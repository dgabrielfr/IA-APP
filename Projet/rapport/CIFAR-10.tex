\chapter{CIFAR-10}


La base CIFAR-10 est une base d'images de petites taille contenant 60000 images de 32 * 32 pixels. Ces images sont réparties dans 10 classes (avions, voiture, oiseau, chat, biche, chien, grenouille, cheval, bateau et camion), chacune contenant 6000 images.

Il y a 6 lots de données répartis en 5 d'apprentissage et 1 de test.

Le lot de test contient 10000 images avec 1000 images sélectionnées dans chaque classe. Les lots d'entrainement contiennent le reste.

Il n'existe pas de recouvrement entre les classes, elles sont mutuellement exclusives.


Les données peuvent être récupérées sous différents format. Dans le cadre d'une utilisation avec \texttt{python}, on utilisera des fichiers produits par cPikle, qui permettent d'obtenir un dictionnaire de la forme :

\begin{itemize}
\item[clé : data] 10000x3072 d'entier non signés 8 bit (0-255). Chaque ligne contient une image 32 * 32 pixels. Les canaux sont ensuite dans l'ordre R (rouge), G (vert), B (bleu), chaque canal a 1024 (32 * 32) éléments. Les images sont stockées par lignes.

\item[clé : label] 10000 contient un nombre de 0 à 9 permettant de connaitre la classe des 10000 images.
\end{itemize}

La page \url{http://rodrigob.github.io/are_we_there_yet/build/classification_datasets_results.html} contient les meilleurs résultats obtenu sur la base CIFAR-10. 

Actuellement le taux de reconnaissance atteint 96.53~\% \cite{DBLP:journals/corr/Graham14a}.

Il est très peu probable que nous atteignons un tel taux de reconnaissance. En effet, ceux-ci suppose des algorithmes complexes avec des réseaux de neurones  comme dans \cite{DBLP:journals/corr/Graham14a}. Le temps limité ne nous permettra pas d'implémenter et de trouver les meilleurs paramètres pour obtenir les meilleurs résultats.

Un autre problème sera la puissance de calcul disponible qui ne permettront pas de traiter un très grand nombre d'images d'apprentissage.


